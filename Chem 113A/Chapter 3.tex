\documentclass[letterpaper]{article}
\usepackage[utf8]{inputenc}
\usepackage{amsmath}
\usepackage{lmodern}
\usepackage[margin=1in]{geometry}
\usepackage{mhchem}
\usepackage{indentfirst}

\newcommand{\diff}{\mathrm{d}}
\newcommand{\zero}{^\circ}

\makeatletter
\begin{document}
\newpage
\section*{Chapter 3}
\subsection*{Entropy}
Entropy is a measure of randomness. The statistical definition of entropy is
\begin{equation*}
    s=k\ln\mathcal{W}
\end{equation*}

The thermodynamic definition is
\begin{equation*}
    \Delta S=\frac{q_{rev}}{T}
\end{equation*}

And $\Delta S$ is a state function

\subsubsection*{The Second Law of Thermodynamics}
The total entropy of a spontaneous process is greater than zero.\\

The definition can be also written in differential form
\begin{equation*}
    \diff S=\frac{\delta q_{rev}}{T}
\end{equation*}

\subsubsection*{Clausius Inequality}
From the first law,
\begin{equation*}
    |\diff w_{rev}|\geq|\diff w|
\end{equation*}

For system doing work, $\diff w_{rev}$ is always negative, so
\begin{equation*}
    -\diff w_{rev}\geq-\diff w
\end{equation*}
\begin{equation*}
    \diff w - \diff w_{rev}\geq0
\end{equation*}

Thus,
\begin{equation*}
    \delta q_{rev}-\delta q\geq0
\end{equation*}
\begin{equation*}
    \frac{\delta q_{rev}}{T}\geq\frac{\delta q}{T}
\end{equation*}

Using the definition of $\diff S$, we can get
\begin{equation*}
    \boxed{\diff S\geq\frac{\delta q}{T}}
\end{equation*}

\subsection*{Maxwell Relation}
Because
\begin{equation*}
    \diff U=\delta w+\delta q=\delta w_{rev}+\delta q_{rev}
\end{equation*}
\begin{equation*}
    \delta w-\delta w_{rev}=\delta q_{rev} - \delta q
\end{equation*}

For reversible changes,
\begin{equation*}
    \delta q_{rev} = T\diff S
\end{equation*}
\begin{equation*}
    \delta w_{rev} = -P\diff V
\end{equation*}

Thus,
\begin{equation*}
    \diff U=T\diff S-P\diff V
\end{equation*}

And we know that,
\begin{equation*}
    \diff U=\left(\frac{\partial U}{\partial S}\right)_V\diff S+\left(\frac{\partial U}{\partial V}\right)_S\diff V
\end{equation*}

By comparing these two expressions, it's obvious that
\begin{equation*}
    T=\left(\frac{\partial U}{\partial S}\right)_V
\end{equation*}
\begin{equation*}
    P=-\left(\frac{\partial U}{\partial V}\right)_S
\end{equation*}

From a property of exact differential that
\begin{equation*}
    \left(\frac{\partial g(x, y)}{\partial y}\right)_x=\left(\frac{\partial h(x, y)}{\partial y}\right)_y
\end{equation*}

We can write similar equation
\begin{equation*}
    \left(\frac{\partial}{\partial V}\left(\frac{\partial U}{\partial S}\right)_V\right)_S=
    \left(\frac{\partial}{\partial S}\left(\frac{\partial U}{\partial V}\right)_S\right)_V
\end{equation*}

So,
\begin{equation*}
    \left(\frac{\partial T}{\partial V}\right)_S=-\left(\frac{\partial P}{\partial S}\right)_V
\end{equation*}
which is one of Maxwell relations

\subsubsection*{Internal Pressure}
We define the internal pressure $\pi_T$ equals to $(\frac{\partial U}{\partial V})_T$.
From the definition $\diff U=T\diff S-P\diff V$, we can get
\begin{equation*}
    \begin{aligned}
        \left(\frac{\partial U}{\partial V}\right)_T & =T\left(\frac{\partial S}{\partial V}\right)-P \\
                                                     & =\left(\frac{\partial U}{\partial S}\right)_V
        \left(\frac{\partial S}{\partial V}\right)_T+
        \left(\frac{\partial U}{\partial V}\right)_S
    \end{aligned}
\end{equation*}

Substituting Maxwell relations again,
\begin{equation*}
    \pi_T=T\left(\frac{\partial P}{\partial T}\right)_V-P
\end{equation*}

\subsection*{Gibbs Free Energy}
The Gibbs free energy is defined as
\begin{equation*}
    G=H-TS=U+PV-TS
\end{equation*}

When taking derivatives,
\begin{equation*}
    \begin{aligned}
        \diff G & =\diff U+P\diff V+V\diff P-T\diff S-S\diff T           \\
                & =T\diff S-P\diff V+P\diff V+V\diff P-T\diff S-S\diff T \\
                & =V\diff P-S\diff T
    \end{aligned}
\end{equation*}

At constant temperature and pressure, $\diff U=\diff H$.
And from Clausius inequality, we can write
\begin{equation*}
    \diff S-\frac{\diff H}{T}\geq0
\end{equation*}

So,
\begin{equation*}
    T\diff S-\diff H\geq0
\end{equation*}
\begin{equation*}
    T\diff S-V\diff P-T\diff S\geq0
\end{equation*}
\begin{equation*}
    \diff G\leq0
\end{equation*}

For a spontaneous process, $\diff G$ is always less than or equal to 0 at constant temperature and pressure

\subsubsection*{Variation of Gibbs Energy with Temperature}
Because
\begin{equation*}
    \diff G=\left(\frac{\partial G}{\partial T}\right)_P\diff T+\left(\frac{\partial G}{\partial P}\right)_T\diff P
\end{equation*}
\begin{equation*}
    V=\left(\frac{\partial G}{\partial P}\right)_T
\end{equation*}
\begin{equation*}
    -S=\left(\frac{\partial G}{\partial T}\right)_P
\end{equation*}

Because $S=\frac{H-G}{T}$, we can get
\begin{equation*}
    \left(\frac{\partial G}{\partial T}\right)_p=\frac{G-H}{T}
\end{equation*}

After several derivation, we can get Gibbs-Helmholtz equation.
\begin{equation*}
    \boxed{\left(\frac{\partial (G/T)}{\partial T}\right)_P=-\frac{H}{T^2}}
\end{equation*}

\subsubsection*{Variation of Gibbs Energy with Pressure}
In this situation, temperature is constant, so
\begin{equation*}
    \diff G=V\diff T
\end{equation*}

When pressure changes from $P\zero$ to $P$, we integrate both sides
\begin{equation*}
    \int_{P\zero}^{P}\diff G=\int_{P\zero}^{P}V\diff P
\end{equation*}

Assuming ideal behavior, which means $PV=nRT$, we can get
\begin{equation*}
    G=G\zero + nRT\ln\frac{P}{P\zero}
\end{equation*}
\begin{equation*}
    G_m=G_m\zero+RT\ln\frac{P}{P\zero}
\end{equation*}
\end{document}