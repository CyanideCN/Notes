\documentclass[letterpaper]{article}
\usepackage[utf8]{inputenc}
\usepackage{amsmath}
\usepackage{lmodern}
\usepackage[margin=1in]{geometry}
\usepackage{mhchem}
\usepackage{indentfirst}

\newcommand{\diff}{\mathrm{d}}
\newcommand{\zero}{^\circ}

\makeatletter
\begin{document}
\newpage
\section*{Chapter 4}
\subsection*{Chemical Potential}
For a system, $\diff G$ can be written as
\begin{equation*}
    \diff G=(\frac{\partial G}{\partial T})_{P,n_i}+(\frac{\partial G}{\partial P})_{T,n_i}
    +\sum_i(\frac{\partial G}{\partial n_i})_{T,P,n_j\neq n_i}
\end{equation*}

At constant pressure and temperature, it can be simplified to
\begin{equation*}
    \diff G=\sum_i(\frac{\partial G}{\partial n_i})_{T,P,n_j\neq n_i}
\end{equation*}

We define each component in $\diff G$ as chemical potential $\mu$ such that
\begin{equation*}
    \mu_i=(\frac{\partial G}{\partial n_i})_{T,P,n_j\neq n_i}
\end{equation*}

For a system containing to components $n_1$ and $n_2$,
\begin{equation*}
    \begin{aligned}
        \diff G&=(\frac{\partial G}{\partial n_1})_{T,P,n_2}\diff n_1+(\frac{\partial G}{\partial n_2})
        _{T,P,n_1}\diff n_2\\
        &=\mu_1\diff n_1+\mu_2\diff n_2
    \end{aligned}
\end{equation*}

When at equilibrium,
\begin{equation*}
    \mu_1\diff n_1=-\mu_2\diff n_2
\end{equation*}
\subsection*{Thermodynamic Aspect - Clapeyron Law}
\begin{equation*}
    \diff\mu=\diff G_m=-S_m\diff T+V_m\diff P
\end{equation*}

Consider phases $\alpha$ and $\beta$, the chemical potential across two phases are equal.
\begin{equation*}
    -S_{m\alpha}\diff T+V_{m\alpha}\diff P=-S_{m\beta}\diff T+V_{m\beta}\diff P
\end{equation*}
\begin{equation*}
    \boxed{\frac{\diff P}{\diff T}=\frac{S_{m\beta}-S_{m\alpha}}{V_{m\beta}-V_{m\alpha}}
    =\frac{\Delta S_{trans}}{\Delta V_{trans}}}
\end{equation*}
\subsubsection*{Solid-Liquid Boundary}
Because the change of entropy when phase changes is $\frac{\Delta H}{T}$, when describing solid-liquid
transformation, the Clapeyron equation can be written as
\begin{equation*}
    \frac{\diff P}{\diff T}=\frac{\Delta_{fus}H}{T\Delta_{fus}V}
\end{equation*}

For a change from $T^*$ to $T$ and $P^*$ to $P$, the path will be
\begin{equation*}
    \int_{P^*}^{P}\diff P=\frac{\Delta_{fus}H}{\Delta_{fus}V}\int_{T^*}^{T}\frac{\diff T}{T}
\end{equation*}
assuming $\frac{\Delta_{fus}H}{\Delta_{fus}V}$ is constant.

Thus, the solid-liquid boundary will be
\begin{equation*}
    P=P^*+\frac{\Delta_{fus}H}{\Delta_{fus}V}\ln\frac{T}{T^*}
\end{equation*}

And $\ln\frac{T}{T^*}$ can be approximated by $\frac{T-T^*}{T^*}$ using Taylor expansion, so the boundary
is
\begin{equation*}
    P\approx P^*+\frac{\Delta_{fus}H}{T^*\Delta_{fus}V}(T-T^*)
\end{equation*}
\subsubsection*{Liquid-Vapor Boundary}
For a similiar approach, the liquid-vapor boundary can be expressed as
\begin{equation*}
    \frac{\diff P}{\diff T}=\frac{\Delta_{vap}H}{T\Delta_{vap}V}
\end{equation*}

We assume that the volume of liquid is negligible, so $\Delta V$ is all from
the volume of gas. By the ideal gas law,
\begin{equation*}
    V_m=\frac{RT}{P}
\end{equation*}

So,
\begin{equation*}
    \frac{\diff P}{\diff T}=\frac{P\Delta_{vap}H}{RT^2}
\end{equation*}

Plugging in $\frac{\diff x}{x}=\diff\ln x$, we get Clausius-Clapeyron equation.
\begin{equation*}
    \boxed{\frac{\diff\ln P}{\diff T}=\frac{\Delta_{vap}H}{RT^2}}
\end{equation*}

When integrating from $T^*$ to $T$ and $P^*$ to $P$, we get
\begin{equation*}
    \ln\frac{P}{P^*}=-\frac{\Delta_{vap}H}{R}(\frac{1}{T}-\frac{1}{T^*})
\end{equation*}
\end{document}