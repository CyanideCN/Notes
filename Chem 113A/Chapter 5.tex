\documentclass[letterpaper]{article}
\usepackage[utf8]{inputenc}
\usepackage{amsmath}
\usepackage{lmodern}
\usepackage[margin=1in]{geometry}
\usepackage{mhchem}
\usepackage{indentfirst}

\newcommand{\diff}{\mathrm{d}}
\newcommand{\zero}{^\circ}

\makeatletter
\begin{document}
\newpage
\section*{Chapter 5}
\subsection*{Partial Molar Quantities}
\subsubsection*{Volume}
The partial molar volume is defined as
\begin{equation*}
    V_j=(\frac{\partial V}{\partial n_j})_{T,P,n_i\neq n_j}
\end{equation*}

For a two-component system,
\begin{equation*}
    \diff V=(\frac{\partial V}{\partial n_A})_{T,P,n_b}\diff n_A
    +(\frac{\partial V}{\partial n_B})_{T,P,n_A}\diff n_B
\end{equation*}
\begin{equation*}
    \diff V=V_A\diff n_A+V_B\diff n_B
\end{equation*}

Thus,
\begin{equation*}
    V=V_An_A+V_Bn_B
\end{equation*}
\subsubsection*{Gibbs Free Energy}
The partial molar Gibbs free energy is defined as chemical potential as discussed before.
So,
\begin{equation*}
    \mu_j=(\frac{\partial G}{\partial n_j})_{T,P,n_i\neq n_j}
\end{equation*}
\begin{equation*}
    G=n_A\mu_A+n_B\mu_B
\end{equation*}
\subsection*{Thermodynamics of Mixing}
\subsubsection*{Mixing of Perfect Gas}
As discussed in Chapter 3,
\begin{equation*}
    G_m=G_m\zero+RT\ln\frac{P}{P\zero}
\end{equation*}

In terms of chemical potential,
\begin{equation*}
    \mu=\mu\zero+RT\ln\frac{P}{P\zero}
\end{equation*}

Because $P_A=x_AP$ where $x$ is the mole fraction of A,
\begin{equation*}
    \mu_A^{mix}=\mu_A^{pure}+RT\ln x_A
\end{equation*}

And because $x_A$ ranges from 0 to 1, $\n x_A$ is never positive, so
\begin{equation*}
    \Delta G_{mix}\leq 0
\end{equation*}

This confirms the mixing process is spontaneous.

Consider the mixing of gas A and gas B. In the initial condition where two gas are separated,
the Gibbs energy is
\begin{equation*}
    G_1=n_A\mu_A+n_B\mu_B
\end{equation*}

And the Gibbs energy for mixture is
\begin{equation*}
    G_2=n_A(\mu_A+RT\ln x_A)+n_B(\mu_B+RT\ln x_B)
\end{equation*}

So, the change of Gibbs energy of mixing is
\begin{equation*}
    \Delta_{mix}G=n_ART\ln x_A+n_BRT\ln x_B
\end{equation*}

In general,
\begin{equation*}
    \Delta_{mix}G=\sum_in_iRT\ln x_i=nRT\sum_ix_i\ln x_i
\end{equation*}
\begin{equation*}
    \Delta_{mix}S=\frac{-\Delta_{mix}G}{T}=-nR\sum_ix_i\ln x_i
\end{equation*}
\subsubsection*{Mixing of Liquid}
For solution, there is an equilibrium of liquid and corresponding gas.
\begin{equation*}
    \mu_A^*(l)=\mu_A\zero(g)+RT\ln\frac{P_A^*}{P\zero}
\end{equation*}

At equilibrium, $\mu_A(l)=\mu_A(g)$, so
\begin{equation*}
    \mu_A\zero(g)=\mu_A^*-RT\ln\frac{P_A}{P\zero}
\end{equation*}
where $P_A^*$ is vapor pressure of pure A.

When another substance B is added,
\begin{equation*}
    \begin{aligned}
        \mu_A(l) & =\mu_A^*(l)-RT\ln\frac{P_A^*}{P\zero}+RT\ln\frac{P_A}{P\zero} \\
                 & =\mu_A^*(l)+RT\ln\frac{P_A}{P_A^*}
    \end{aligned}
\end{equation*}

From the Raolt's law where $P_A=x_AP_A^*$, we can write
\begin{equation*}
    \mu_A(l)=\mu_A^*(l)+RT\ln x_A
\end{equation*}

However, when solution is dilute, $P_A\propto x_A$, which is known as Henry's law.

\subsection*{Colligative Properties}
\end{document}