\documentclass[letterpaper]{article}
\usepackage[utf8]{inputenc}
\usepackage{amsmath}
\usepackage{lmodern}
\usepackage[margin=1in]{geometry}
\usepackage{mhchem}
\usepackage{indentfirst}

\newcommand{\diff}{\mathrm{d}}
\newcommand{\zero}{^\circ}

\makeatletter
\begin{document}
\newpage
\section*{Chapter 5}
\subsection*{Partial Molar Quantities}
\subsubsection*{Volume}
The partial molar volume is defined as
\begin{equation*}
    V_j=\left(\frac{\partial V}{\partial n_j}\right)_{T,P,n_i\neq n_j}
\end{equation*}

For a two-component system,
\begin{equation*}
    \diff V=\left(\frac{\partial V}{\partial n_A}\right)_{T,P,n_b}\diff n_A
    +\left(\frac{\partial V}{\partial n_B}\right)_{T,P,n_A}\diff n_B
\end{equation*}
\begin{equation*}
    \diff V=V_A\diff n_A+V_B\diff n_B
\end{equation*}

Thus,
\begin{equation*}
    V=V_An_A+V_Bn_B
\end{equation*}
\subsubsection*{Gibbs Free Energy}
The partial molar Gibbs free energy is defined as chemical potential as discussed before.
So,
\begin{equation*}
    \mu_j=\left(\frac{\partial G}{\partial n_j}\right)_{T,P,n_i\neq n_j}
\end{equation*}
\begin{equation*}
    G=n_A\mu_A+n_B\mu_B
\end{equation*}
\subsection*{Thermodynamics of Mixing}
\subsubsection*{Mixing of Perfect Gas}
As discussed in Chapter 3,
\begin{equation*}
    G_m=G_m\zero+RT\ln\frac{P}{P\zero}
\end{equation*}

In terms of chemical potential,
\begin{equation*}
    \mu=\mu\zero+RT\ln\frac{P}{P\zero}
\end{equation*}

Because $P_A=x_AP$ where $x$ is the mole fraction of A,
\begin{equation*}
    \mu_A^{mix}=\mu_A^{pure}+RT\ln x_A
\end{equation*}

And because $x_A$ ranges from 0 to 1, $\ln x_A$ is never positive, so
\begin{equation*}
    \Delta G_{mix}\leq 0
\end{equation*}

This confirms the mixing process is spontaneous.

Consider the mixing of gas A and gas B. In the initial condition where two gas are separated,
the Gibbs energy is
\begin{equation*}
    G_1=n_A\mu_A+n_B\mu_B
\end{equation*}

And the Gibbs energy for mixture is
\begin{equation*}
    G_2=n_A\left(\mu_A+RT\ln x_A\right)+n_B\left(\mu_B+RT\ln x_B\right)
\end{equation*}

So, the change of Gibbs energy of mixing is
\begin{equation*}
    \Delta_{mix}G=n_ART\ln x_A+n_BRT\ln x_B
\end{equation*}

In general,
\begin{equation*}
    \Delta_{mix}G=\sum_in_iRT\ln x_i=nRT\sum_ix_i\ln x_i
\end{equation*}
\begin{equation*}
    \Delta_{mix}S=\frac{-\Delta_{mix}G}{T}=-nR\sum_ix_i\ln x_i
\end{equation*}
\subsubsection*{Mixing of Liquid}
For solution, there is an equilibrium of liquid and corresponding gas.
\begin{equation*}
    \mu_A^*(l)=\mu_A\zero\left(g\right)+RT\ln\frac{P_A^*}{P\zero}
\end{equation*}

At equilibrium, $\mu_A(l)=\mu_A\left(g\right)$, so
\begin{equation*}
    \mu_A\zero\left(g\right)=\mu_A^*-RT\ln\frac{P_A}{P\zero}
\end{equation*}
where $P_A^*$ is vapor pressure of pure A.

When another substance B is added,
\begin{equation*}
    \begin{aligned}
        \mu_A(l) & =\mu_A^*(l)-RT\ln\frac{P_A^*}{P\zero}+RT\ln\frac{P_A}{P\zero} \\
                 & =\mu_A^*(l)+RT\ln\frac{P_A}{P_A^*}
    \end{aligned}
\end{equation*}

From the Raolt's law where $P_A=x_AP_A^*$, we can write
\begin{equation*}
    \mu_A(l)=\mu_A^*(l)+RT\ln x_A
\end{equation*}

However, when solution is dilute, $P_A\propto x_A$, which is known as Henry's law.

\subsection*{Colligative Properties}


\subsection*{Osmotic Pressure}
Consider a container separated by a semipermeable membrane. The left side contains pure A
and the right side contains both A and B. The pressure above left side is $P$ and the pressure
above right side is $P+\Pi$

At equilibrium, chemical potentials at both sides are equal.
\begin{equation*}
    \mu_A^*(P)=\mu_A\left(x_A,P+\Pi\right)
\end{equation*}
\begin{equation*}
    \mu_A^*\left(x_A,P+\Pi\right)=\mu_A^*(P)-RT\ln x_A
\end{equation*}

At constant temperature, $\diff G_m=V_m\diff P$, so
\begin{equation*}
    G_{m,f}=G_{m,i}+\int_{P_i}^{P_f}V_m\diff P
\end{equation*}
\begin{equation*}
    \mu_A^*\left(x_A,P+\Pi\right)=\mu_A^*(P)+\int_{P}^{P+\Pi}V_m\diff P
\end{equation*}

And this gives
\begin{equation*}
    \mu_A^*(P)+\int_{P}^{P+\Pi}V_m\diff P=\mu_A^*(P)-RT\ln x_A
\end{equation*}
\begin{equation*}
    -RT\ln x_A=\int_{P}^{P+\Pi}V_m\diff P
\end{equation*}

When the osmotic pressure $\Pi$ is much smaller relative to $P$, this simplifies to
\begin{equation*}
    -RT\ln x_A=V_m\left(P+\Pi\right)-V_mP=V_m\Pi
\end{equation*}

And when $x_B$ is very small, $\ln x_A=\ln\left(1-x_B\right)\approx-x_B$, we can write the osmotic
pressure as
\begin{equation*}
    \Pi=\frac{RTx_B}{V_m}=RT\frac{n_B}{nV_m}=RT\frac{n_B}{V}
\end{equation*}
\begin{equation*}
    \boxed{\Pi=RTc_B}
\end{equation*}
\subsection*{Activity}
\subsubsection*{Solvent Activity}
We know that
\begin{equation*}
    \mu_A=\mu_A^*+RT\ln x_A
\end{equation*}

The activity of A is defined as
\begin{equation*}
    \mu_A=\mu_A^*+RT\ln a_A
\end{equation*}

And $a_A$ approaches $x_A$ when $x_A$ approaches to 1. We use a new coefficient to describe that
\begin{equation*}
    a_A=\gamma_Ax_A
\end{equation*}

$\gamma_A$ approaches 1 when $x_A$ approaches to 1.

The corrected Rault's law with activity is
\begin{equation*}
    \mu_A=\mu_A^*+RT\ln a_A\gamma_A=\mu_A^*+RT\ln a_A+RT\ln\gamma_A
\end{equation*}
\subsubsection*{Solute Activity}
For ideal situation where $x_B$ approaches to zero, Henry's law states that
\begin{equation*}
    P_B=K_Bx_B
\end{equation*}

The chemical potential is
\begin{equation*}
    \begin{aligned}
        \mu_B & =\mu_B^*+RT\ln\frac{P_B}{P_B^*}         \\
              & =\mu_B^*+RT\ln\frac{K_Bx_B}{P_B^*}      \\
              & =\mu_B^*+RT\ln\frac{K_B}{P_B^*}+\ln x_B
    \end{aligned}
\end{equation*}

The term $\mu_B^*+RT\ln\frac{K_B}{P_B^*}$ is constant and can be replaced by a
new constant $\mu_B\zero$. The expression then comes back to ideal equation.

For real solutes, we use similar approach.
\begin{equation*}
    \mu_B=\mu_B\zero+RT\ln a_B
\end{equation*}
\end{document}