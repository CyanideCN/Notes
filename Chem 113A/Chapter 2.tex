\documentclass[letterpaper]{article}
\usepackage[utf8]{inputenc}
\usepackage{amsmath}
\usepackage{lmodern}
\usepackage[margin=1in]{geometry}
\usepackage{mhchem}
\usepackage{indentfirst}

\newcommand{\diff}{\mathrm{d}}

\makeatletter
\begin{document}
\newpage
\section*{Chapter 2}
\subsection*{Internal Energy}
The internal energy of an isolated system is conserved.
\begin{equation*}
    \begin{aligned}
        &\Delta U(\mathrm{total})=\Delta U(\mathrm{system})+\Delta U(\mathrm{surrounding})\\
        &\Delta U(\mathrm{system})=-\Delta U(\mathrm{surrounding})
    \end{aligned}
\end{equation*}

Also,
\begin{equation*}
    \Delta U=w+q
\end{equation*}
\begin{equation*}
    \diff U=\delta w+\delta q
\end{equation*}

The work $w$ refers to mechanical work, or PV work.
\subsubsection*{Isochoric Process}
Consider an isochoric process,
\begin{equation*}
    \diff U=\delta q + \delta w = \delta q - P\diff V
\end{equation*}

because the volume doesn't change, d$V=0$,
\begin{equation*}
    \diff U_V=\delta q_V
\end{equation*}

because $C_V=\frac{\delta q_V}{\delta T}=\frac{\delta q_V}{\diff T}$,
\begin{equation*}
    \diff U_V=C_V\diff T
\end{equation*}
\begin{equation*}
    \boxed{C_V=(\frac{\partial U}{\partial T})_V}
\end{equation*}
\subsubsection*{Isobaric Process}
We define a new variable enthalpy (H).
\begin{equation*}
    H=U+PV
\end{equation*}

Differentiating both sides,
\begin{equation*}
    \begin{aligned}
        \diff H&=\diff (U+PV)\\
        &=\diff U+\diff (PV)\\
        &=\diff U+P\diff V+V\diff P
    \end{aligned}
\end{equation*}

Plugging in $\diff U=\delta q-P\diff V$,
\begin{equation*}
    \begin{aligned}
        \diff H&=(\delta q-P\diff V)+P\diff V+V\diff P\\
        &=\delta q+V\diff P
    \end{aligned}
\end{equation*}

Under isobaric condition, where $\diff P=0$,
\begin{equation*}
    \begin{aligned}
        \diff H_P&=\delta q_P=\frac{\delta q_P}{\diff T}\diff T\\
        &=C_P\diff T
    \end{aligned}
\end{equation*}

Thus,
\begin{equation*}
    \boxed{C_P=(\frac{\partial H}{\partial T})_P}
\end{equation*}

For reversible isobaric expansion,
\begin{equation*}
    w=-P_{ext}\int_{V_1}^{V_2}\diff V=P_{ext}\Delta V
\end{equation*}
\subsubsection*{Relationship between $C_P$ and $C_V$}
From derivation in isobaric process, we know
\begin{equation*}
    C_P\diff T=\diff U+P\diff V
\end{equation*}

And from derivation in isochoric process, we know
\begin{equation*}
    \diff U=C_V\diff T
\end{equation*}

Thus,
\begin{equation*}
    C_P\diff T=C_V\diff T+P\diff V
\end{equation*}

By differentiating ideal gas law,
\begin{equation*}
    P\diff V+V\diff P=nR\diff T
\end{equation*}

At constant pressure, this becomes
\begin{equation*}
    P\diff V=nR\diff T
\end{equation*}

Then,
\begin{equation*}
    C_P\diff T=C_V\diff T+nR\diff T
\end{equation*}
\begin{equation*}
    \boxed{C_P=C_V+nR}
\end{equation*}

For monatomic ideal gas,
\begin{equation*}
    C_V=(\frac{\partial U}{\partial T})_V=\frac{3}{2}nR
\end{equation*}
\begin{equation*}
    C_P=C_V+nR=\frac{5}{2}nR
\end{equation*}

Or in terms of molar heat capacity,
\begin{equation*}
    c_P=c_V+R
\end{equation*}
\subsubsection*{Isothermal Process}
For reversible isothermal expansion,
\begin{equation*}
    w=-\int_{V_1}^{V_2}P\diff V=-nRT\int_{V_1}^{V_2}\frac{\diff V}{V}=-nRT\ln\frac{V_2}{V_1}
\end{equation*}
\subsubsection*{Adiabatic Process}
Adiabatic condition implies $\delta q=0$
\begin{equation*}
    \begin{aligned}
        \diff U_ad&=\delta q_ad + \delta w_ad\\
        &=-p\diff V=C_V\diff T
    \end{aligned}
\end{equation*}

Thus,
\begin{equation*}
    \begin{aligned}
        C_V\diff T&=-P\diff V=-\frac{nRT}{V}\diff V\\
        C_V\frac{\diff T}{T}&=-nR\frac{\diff V}{V}
    \end{aligned}
\end{equation*}

Considering the initial condition $V_i$ and $T_i$, the final condition $V_f$ and $T_f$,
\begin{equation*}
    C_V\int_{T_i}^{T_f}\frac{\diff T}{T}=-nR\int_{V_i}^{V_f}\frac{\diff V}{V}
\end{equation*}
\begin{equation*}
    C_V\ln\frac{T_f}{T_i}=-nR\ln\frac{V_f}{V_i}=-(C_P-C_V)\ln\frac{V_f}{V_i}
\end{equation*}
\begin{equation*}
    \ln\frac{T_f}{T_i}=-(\frac{C_P}{C_V}-1)\ln\frac{V_f}{V_i}
\end{equation*}

We define $\gamma=\frac{C_P}{C_V}=\frac{c_P}{c_V}$,
\begin{equation*}
    \ln\frac{T_f}{T_i}=-(\gamma-1)\ln\frac{V_f}{V_i}
\end{equation*}
\begin{equation*}
    \ln\frac{T_i}{T_f}=\ln(\frac{V_f}{V_i})^{-(\gamma-1)}
\end{equation*}
\begin{equation*}
    \frac{T_f}{T_i}=(\frac{V_f}{V_i})^{-(\gamma-1)}=(\frac{V_i}{V_f})^{\gamma-1}
\end{equation*}
\begin{equation*}
    \frac{P_fV_f}{P_iV_i}=\frac{T_f}{T_i}=(\frac{V_f}{V_i})^{-(\gamma-1)}
\end{equation*}
\begin{equation*}
    \boxed{\frac{P_f}{P_i}=(\frac{V_f}{V_i})^{-\gamma}=(\frac{V_i}{V_f})^\gamma}
\end{equation*}
\subsection*{Thermochemistry}
For a chemical reaction at standard condition,
\begin{equation*}
    \nu_AA+\nu_BB\rightleftharpoons\nu_CC+\nu_DD
\end{equation*}

The enthalpy change of reaction is
\begin{equation*}
    \Delta_rH^\circ=\sum_P\nu_P\Delta_fH^\circ_P-\sum_R\nu_R\Delta_fH^\circ_R
\end{equation*}

\subsubsection*{Temperature Dependence}
Because $\diff H=C_P\diff T$, we can write
\begin{equation*}
    H(T_2)=H(T_1)+\int_{T_1}^{T_2}C_P\diff T
\end{equation*}

Applying it to the enthalpy change of reaction yield Kirchhoff’s law.
\begin{equation*}
    \Delta_rH(T_2)=\Delta_rH(T_1)+\int_{T_1}^{T_2}\Delta_rC_P\diff T
\end{equation*}

If $C_P$ is independent of temperature, Kirchhoff’s law can be simplified to
\begin{equation*}
    \Delta_rH(T_2)=\Delta_rH(T_1)+\Delta_rC_P(T_2-T_1)
\end{equation*}
\subsection*{Joule–Thomson Effect}
The work is composed of work from upstream and downstream gases, denoted as $w_1$ and $w_2$.
\begin{equation*}
    w_1=-P_1(0-V_1)=P_1V_1
\end{equation*}
\begin{equation*}
    w_2=-P_2(V_2-0)=-P_2V_2
\end{equation*}

Thus,
\begin{equation*}
    w=P_1V_1-P_2V_2
\end{equation*}

Because no heat is transferred ($q=0$), the change of internal energy equals to work.
\begin{equation*}
    \Delta U=U_2-U_1=P_1V_1-P_2V_2
\end{equation*}
\begin{equation*}
    U_2+P_2V_2=U_1+P_1V_1
\end{equation*}
And this yields $H_1=H_2$.

We know the equation
\begin{equation*}
    \diff H=(\frac{\partial H}{\partial T})_P\diff T+(\frac{\partial H}{\partial P})_T\diff P
\end{equation*}

And because the process is isoenthalpic, which means $\Delta H=0$
\begin{equation*}
    0=C_P\diff T+(\frac{\partial H}{\partial P})_T\diff P
\end{equation*}

Then, rearrange this equation
\begin{equation*}
    (\frac{\partial H}{\partial P})_T=-C_P(\frac{\partial T}{\partial P})_H
\end{equation*}

We define the Joule–Thomson coefficient $\mu$ as
\begin{equation*}
    \mu=(\frac{\partial T}{\partial P})_H
\end{equation*}

So the change of enthalpy with respect to pressure can be written as
\begin{equation*}
    \boxed{(\frac{\partial H}{\partial P})_T=-C_P\mu}
\end{equation*}
\end{document}