\documentclass[letterpaper]{article}
\usepackage[utf8]{inputenc}
\usepackage{amsmath}
\usepackage{lmodern}
\usepackage[margin=1in]{geometry}

\makeatletter
\begin{document}
\newpage
\section*{Chapter 1}
\subsection*{Boltzmann Distribution}
\begin{equation*}
    N_i=\frac{N_0 g_i e^{-\epsilon_i/kT}}{\Sigma g_i e^{-\epsilon_i/kT}}
\end{equation*}
where $g_i$ is degeneracy of state $i$.
And the differential probability would be
\begin{equation*}
    dP_{\epsilon t} = \frac{dn_{\epsilon t}}{N} = \frac{g_{\epsilon i} e^{-\epsilon i/kT}
    d\epsilon_t}{\int g_{\epsilon i} e^{-\epsilon i/kT}d\epsilon_t}
\end{equation*}
\subsubsection*{Equipartition Principle}
Every degree of freedom that appears only quadratically in the total energy has an average 
energy of $\frac{1}{2}kT$ in thermal equilibrium.
\begin{equation*}
    \begin{aligned}
        \epsilon _t &= \epsilon _t(x) + \epsilon _t(y) + \epsilon _t(z) \\
        &= \frac{1}{2}mv_x^2 + \frac{1}{2}mv_y^2 + \frac{1}{2}mv_z^2 \\
        &= \frac{3}{2}kT
    \end{aligned}
\end{equation*}
\subsubsection*{Velocity Distribution in the x-direction}
\begin{equation*}
    \begin{aligned}
        dP_{vx} &= \frac{dn_{vx}}{N} = \frac{e^{-\frac{1}{2}mv_x^2}/kT}{\int_{-\infty}^{\infty}
        e^{-\frac{1}{2}mv_x^2/kT}dv_x}
    \end{aligned}
\end{equation*}
Applying the integral value
\begin{equation*}
    I = \int_{-\infty}^{\infty}e^{-ax^2}dx=\sqrt{\frac{\pi}{a}}
\end{equation*}
The original integral becomes
\begin{equation*}
    \begin{aligned}
        I &= \int_{-\infty}^{\infty}e^{-\frac{1}{2}mv_x^2/kT}dv_x\\
        &= \sqrt{\frac{2kT\pi}{m}}
    \end{aligned}
\end{equation*}
Plugging the value back to the differential probability.
\begin{equation*}
    \begin{aligned}
        dP_{vx}=\sqrt{\frac{m}{2kT\pi}}e^{-\frac{mv_x^2}{kT}}dv_x
    \end{aligned}
\end{equation*}
\begin{equation*}
    f(v_x)=\sqrt{\frac{m}{2kT\pi}}e^{-\frac{mv_x^2}{kT}}
\end{equation*}
\subsubsection*{Velocity Distribution in Space}
\begin{equation*}
    \begin{aligned}
        dP_v&=dP_{vx}dP_{vy}dP_{vz}\\
        &=\frac{e^{-\frac{mv_x^2}{2kT}}e^{-\frac{mv_y^2}{2kT}}e^{-\frac{mv_z^2}{2kT}}}
        {\int_{vx}...\int_{vy}...\int_{vz}...}\\
        &=(\frac{m}{2\pi kT})^{\frac{3}{2}}e^{\frac{1}{2kT}m(v_x^2+v_y^2+v_z^2)}
        dv_xdv_ydv_z
    \end{aligned}
\end{equation*}
The velocity components can be expressed as
\begin{equation*}
    \begin{aligned}
        &v_x=v\sin\theta\cos\theta\\
        &v_y=v\sin\theta\sin\theta\\
        &v_z=v\cos\theta
    \end{aligned}
\end{equation*}
Using chain rule, we can get
\begin{equation*}
    dv_xdv_ydv_z=v^2dv\sin\theta d\theta d\Phi
\end{equation*}
\begin{equation*}
    \begin{aligned}
        dP_v=(\frac{m}{2\pi kT})^{\frac{3}{2}}e^{-\frac{mv^2}{2kT}}v^2dv\sin\theta d\theta d\Phi
    \end{aligned}
\end{equation*}
Integrate with respect to $\theta$ and $\Phi$
\begin{equation*}
    \int_{0}^{2\pi}\int_{0}^{\pi}\sin\theta d\theta d\Phi=4\pi
\end{equation*}
\begin{equation*}
    dP_v=4\pi(\frac{m}{2\pi kT})^{\frac{3}{2}}e^{-\frac{mv^2}{2kT}}dv
\end{equation*}
Then we can get the distribution
\begin{equation*}
    f(v)=4\pi v^2(\frac{m}{2\pi kT})^{\frac{3}{2}}e^{-\frac{mv^2}{2kT}}
\end{equation*}
\subsubsection*{Most Probable Value of Velocity Distribution}
The most probable value comes when $\frac{\partial f(v)}{\partial v}=0$
\begin{equation*}
    \begin{aligned}
        \frac{\partial f(v)}{\partial v}&=\frac{\partial}{\partial v}
        [4\pi v^2(\frac{m}{2\pi kT})^{\frac{3}{2}}e^{-\frac{mv^2}{2kT}}]\\
        &=2v\times 4\pi(\frac{m}{2\pi kT})^{\frac{3}{2}}e^{-\frac{mv^2}{2kT}}+
        4\pi v^2(\frac{m}{2\pi kT})^{\frac{3}{2}}(-\frac{mv}{kT})e^{-\frac{mv^2}{2kT}}
        =0
    \end{aligned}
\end{equation*}
Solving the equation
\begin{equation*}
    \begin{aligned}
        2v\times 4\pi + v^2 4\pi(-\frac{mv}{kT})&=0\\
        2-v^2(\frac{m}{kT})&=0
    \end{aligned}
\end{equation*}
\begin{equation*}
    v_{mp}=\sqrt{\frac{2kT}{m}}
\end{equation*}
\subsubsection*{Average Speed}
The expected value of speed is the sum of all probabilities time corresponding
speed values over 0 to $\infty$.
\begin{equation*}
    \begin{aligned}
        \bar{v}&=\int_0^\infty vdP_v=\int_0^\infty vf(v)dv\\
        &=4\pi(\frac{m}{2\pi kT})^\frac{3}{2}\int_0^\infty v^3 e^{-\frac{mv^2}{2kT}}dv
    \end{aligned}
\end{equation*}
Using the integral that
\begin{equation*}
    I=\int_0^\infty x^{2n+1}e^{-ax^2}dx=\frac{n!}{2a^{n+1}}
\end{equation*}
The expression becomes
\begin{equation*}
    \begin{aligned}
        \bar{v}&=4\pi(\frac{m}{2\pi kT})^\frac{3}{2}\times \frac{1}{2}(\frac{2kT}{m})^2\\
        &=\sqrt{\frac{8kT}{\pi m}}
    \end{aligned}
\end{equation*}
\subsubsection*{Velocity Predicted by Equipartition Theorem}
\begin{equation*}
    \begin{aligned}
        &\frac{1}{2}mv^2=\frac{3}{2}kT\\
        &v_{ep}=\sqrt{\frac{3kT}{m}}
    \end{aligned}
\end{equation*}
\subsection*{Collisions}
There are two types of collisions. One is collisions with other molecules, which is related
to a concept called mean free path. The other one is collisions with wall, which is related
to pressure.
\subsubsection*{Mean Free Path}
For two molecules with average speed $\bar{v_1}$ and $\bar{v_2}$, the relative speed $\bar{v_{12}}$ is
\begin{equation*}
    \bar{v_{12}}=\bar{v_1}^2+\bar{v_2}^2-2\bar{v_1}\bar{v_2}\int_0^{2\pi}\cos\theta d\theta=\bar{v_1}^2+\bar{v_2}^2
\end{equation*}
The cross-section area of collision is
\begin{equation*}
    \sigma=\pi d^2
\end{equation*}
The frequency of collision is
\begin{equation*}
    z=\sigma\bar{v_{12}}n=\sqrt{2}\bar{v}d^2\pi n
\end{equation*}
The mean free path will be mean speed divided by frequency of collision
\begin{equation*}
    \lambda=\frac{\bar{v}}{z}=\frac{1}{\sqrt{2}\pi d^2n}
\end{equation*}
\subsubsection*{Pressure}
The basic definition of pressure is force per unit area. And the force can be expressed as change of momentum.
For $j$th molecule, the force in x-direction is
\begin{equation*}
    \bar{F_{xj}}=\frac{\partial}{\partial t}m\bar{v_{xj}}
\end{equation*}
We assume that in $\partial t$, the molecule travels distance of $2a$, and the momentum change is $2m\bar{v_{xj}}$.
The force can then be expressed as
\begin{equation*}
    \bar{F_{xj}}=2m\bar{v_{xj}}\frac{\bar{v_{xj}}}{2a}=\frac{m\bar{v_{xj}}^2}{a}
\end{equation*}
Thus, the pressure expression is
\begin{equation*}
    P_{xj}=\frac{\bar{F_{xj}}}{A}=\frac{m\bar{v_{xj}}^2}{Aa}=\frac{m\bar{v_{xj}}^2}{V}
\end{equation*}
Considering all directions, $\bar{v}^2$ will be 3 times $\bar{v_x}^2$ because there are 3 directions in space.
\begin{equation*}
    P=\sum_{n=1}^{N}P_j=\sum_{n=1}^{N}\frac{1}{3}\frac{m\bar{v}^2}{V}=\frac{N}{3}\frac{m\bar{v}^2}{V}
\end{equation*}
Because of equipartition theorem that $mv^2=3kT$, thus
\begin{equation*}
    \begin{aligned}
        &P=\frac{NkT}{V}\\
        &PV=NkT
    \end{aligned}
\end{equation*}
When using gas contant, the expression becomes ideal gas equation
\begin{equation*}
    PV=nRT
\end{equation*}
where $k=\frac{R}{N_A}$.
\subsection*{Real Gas}
Compared to ideal gas, real molecules have some different properties. Real gas molecules have volume, and the collision
between them is inelastic, these molecules also interact with each other.
For real gas, the compression factor $z$ is given by
\begin{equation*}
    z=\frac{V_r}{V_\circ}
\end{equation*}
where $V_r$ is real volume and $V_\circ$ is predicted volume.\\
When $z$ is smaller than 1, the attraction between molecules dominates. When $z$ is greater than 1, the repulsion dominates.
\subsubsection*{Virial Equation for Real Gases}
The ideal gas equation can be rearranged to
\begin{equation*}
    P=\frac{RT}{V_m}
\end{equation*}
where $V_m$ is the molar volume.\\
We use series expansion to approximate the behavior of real gases.
\begin{equation*}
    P=\frac{RT}{V_m}(1 + \frac{B'P}{V_m} + \frac{C'P^2}{V_m^2} ...)
\end{equation*}
As pressure is proportional to temperature, $P$ in the expansion terms can be rewritten as function of temperature.
\begin{equation*}
    P=\frac{RT}{V_m}(1+\frac{B(T)}{V_m}+\frac{C(T)}{V_m}...)
\end{equation*}
\subsubsection*{Van der Waals Correction}
Van der Waals proposed two correction temrs to the original equation.
\begin{equation*}
    \begin{aligned}
        P&=\frac{nRT}{V-nb}-\frac{an^2}{V^2}\\
        &=\frac{RT}{V_m-b}-\frac{a}{V_m^2}\\
        &=\frac{RT}{V_m}(1-\frac{b}{V_m})^{-1}-\frac{a}{V_m^2}
    \end{aligned}
\end{equation*}
Using the expansion of $(1-x)^{-1}$
\begin{equation*}
    (1-x)^{-1}=1+x+x^2+...
\end{equation*}
\begin{equation*}
    P=\frac{RT}{V_m}(1+\frac{b-\frac{a}{RT}}{V_m}+\frac{b^2}{V_m^2}+...)
\end{equation*}
\end{document}