\documentclass[letterpaper]{article}
\usepackage[utf8]{inputenc}
\usepackage{amsmath}
\usepackage{lmodern}
\usepackage[margin=1in]{geometry}
\usepackage{mhchem}
\usepackage{indentfirst}

\newcommand{\diff}{\mathrm{d}}
\newcommand{\zero}{^\circ}

\makeatletter
\begin{document}
\newpage
\section*{Chapter 6}
\subsection*{Equilibrium Constant}
For a chemical reaction, the change of Gibbs energy is
\begin{equation*}
    \Delta_rG=\sum_p(\mu_p\zero+RT\ln\frac{P_p}{P\zero})-\sum_r(\mu_r\zero+RT\ln\frac{P_r}{P\zero})
\end{equation*}
\begin{equation*}
    \Delta_rG=\Delta_rG\zero+RT\ln\frac{\prod_p\left(\frac{P_p}{P_0}\right)^{\nu_p}}{\prod_r\left(\frac{P_r}{P_0}\right)^{\nu_r}}
\end{equation*}

We define the reaction quotient $Q$ as $\frac{\prod_p\left(\frac{P_p}{P_0}\right)^{\nu_p}}{\prod_r\left(\frac{P_r}{P_0}\right)^{\nu_r}}$
, so
\begin{equation*}
    \Delta_rG=\Delta_rG\zero+RT\ln Q
\end{equation*}

When at equilibrium, $\Delta_rG=0$, so $\Delta_rG\zero=-RT\ln Q_{eq}$.
\begin{equation*}
    \Delta_rG\zero=\frac{\prod_p\left(\frac{P_p^{eq}}{P_0}\right)^{\nu_p}}
    {\prod_r\left(\frac{P_r^{eq}}{P_0}\right.)^{\nu_r}}(P^\circ)^{\Delta\nu}
\end{equation*}

$\Delta\nu$ is the change of stoichiometric coefficient in the reaction that
\begin{equation*}
    \Delta\nu = \sum_p\nu_p-\sum_r\nu_r
\end{equation*}

By convention, we define the equilibrium constant as
\begin{equation*}
    K_p=\frac{\prod_p\left(\frac{P_p^{eq}}{P_0}\right)^{\nu_p}}
    {\prod_r\left(\frac{P_r^{eq}}{P_0}\right)^{\nu_r}}(P^\circ)^{\Delta\nu}
\end{equation*}
\subsection*{Variation of Equilibrium Constant with Temperature}
We know that
\begin{equation*}
    \ln K_p=-\frac{\Delta_rG\zero}{RT}
\end{equation*}

Differetiating the equation gives
\begin{equation*}
    \frac{\partial\ln K_p}{\partial T}=-\frac{1}{R}\left(\frac{\Delta_rG\zero/T}{\partial T}\right)
\end{equation*}

And the Gibbs-Helmholtz equation shows that
\begin{equation*}
    \frac{\Delta_rG\zero/T}{\partial T}=-\frac{\Delta_rH\zero}{T^2}
\end{equation*}

So, we get
\begin{equation*}
    \frac{\partial\ln K_p}{\partial T}=-\frac{\Delta_rH\zero}{T^2}
\end{equation*}
\begin{equation*}
    \boxed{\frac{\diff\ln K_p}{\diff T}=\frac{\Delta_rH\zero}{RT^2}}
\end{equation*}

This is called van't Hoff equation.

For the change from $T\zero$ to $T$, we integrate the equation that
\begin{equation*}
    \ln K_p(T)-\ln K_p(T\zero)=\int_{T\zero}^{T}\frac{\Delta_rH\zero}{RT^2}
\end{equation*}
\begin{equation*}
    \boxed{\ln K_p(T)=\ln K_p(T\zero)-\frac{\Delta_rH\zero}{R}\left(\frac{1}{T}-\frac{1}{T\zero}\right)}
\end{equation*}
\subsection*{Non-PV Work and Equilibrium}
\subsubsection*{Nernst Equation}
The work done by an electrochemical cell is given by
\begin{equation*}
    w_e=|e(\phi_2-\phi_1)|=|FE|
\end{equation*}
where $E$ is the potential difference across cells and $F$ is Faraday's constant.

Because $\diff G=\Delta_rG\diff \xi$, at constant pressure and temperature,
\begin{equation*}
    \diff G=\diff w_e=|nFE|\diff\xi
\end{equation*}

Thus,
\begin{equation*}
    \Delta_rG=|nFE|
\end{equation*}

Combining the derived Gibss energy equation with reaction quotient, we get
\begin{equation*}
    \Delta_rG\zero+RT\ln Q=|nFE|
\end{equation*}

Because the work is done by system, the sign of work is negative,
\begin{equation*}
    \Delta_rG\zero+RT\ln Q=-nFE
\end{equation*}
\begin{equation*}
    E_{cell}=-\frac{\Delta_rG\zero}{nF}-\frac{RT}{nF}\ln Q
\end{equation*}

And we define the standard cell potential $E\zero$ is
\begin{equation*}
    E_{cell}\zero=-\frac{\Delta_rG}{nF}
\end{equation*}

Then, we get the Nernst equation
\begin{equation*}
    \boxed{E_{cell}=E_{cell}\zero-\frac{RT}{nF}\ln Q}
\end{equation*}
\subsubsection*{Cells at Equilibrium}
When cells are in equilibrium, they can not do any work, so $\Delta_rG=0$. The Nernst equation
becomes
\begin{equation*}
    E_{cell}\zero=\frac{RT}{nF}\ln K
\end{equation*}
\subsubsection*{Thermodynamic Functions and Cell Potential}
We can further rearrange the Nerst equation at equilibrium that
\begin{equation*}
    \ln K=\frac{nFE_{cell}\zero}{RT}
\end{equation*}

Because $\left(\frac{\partial G}{\partial T}_P=-S\right)$,
\begin{equation*}
    \left(\frac{\partial \Delta_rG}{\partial T}\right)\zero_P=-\Delta_rS\zero
\end{equation*}
\begin{equation*}
    \Delta_rS\zero=-nF\left(\frac{\partial E_{cell}\zero}{\partial T}\right)
\end{equation*}

And we can further derive enthalpy.
\begin{equation*}
    \Delta_rH\zero=-nF(E_{cell}\zero-T\frac{\partial E_{cell}\zero}{\partial T})
\end{equation*}
\end{document}