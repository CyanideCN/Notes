\documentclass[letterpaper]{article}
\usepackage[utf8]{inputenc}
\usepackage{amsmath}
\usepackage{lmodern}
\usepackage[margin=1in]{geometry}
\usepackage{mhchem}
\usepackage{indentfirst}

\newcommand{\diff}{\mathrm{d}}
\newcommand{\zero}{^\circ}

\makeatletter
\begin{document}
\newpage
\section*{Chapter 6}
\subsection*{Equilibrium Constant}
For a chemical reaction, the change of Gibbs energy is
\begin{equation*}
    \Delta_rG=\sum_p(\mu_p\zero+RT\ln\frac{P_p}{P\zero})-\sum_r(\mu_r\zero+RT\ln\frac{P_r}{P\zero})
\end{equation*}
\begin{equation*}
    \Delta_rG=\Delta_rG\zero+RT\ln\frac{\prod_p(\frac{P_p}{P_0})^{\nu_p}}{\prod_r(\frac{P_r}{P_0})^{\nu_r}}
\end{equation*}

We define the reaction quotient $Q$ as $\frac{\prod_p(\frac{P_p}{P_0})^{\nu_p}}{\prod_r(\frac{P_r}{P_0})^{\nu_r}}$
, so
\begin{equation*}
    \Delta_rG=\Delta_rG\zero+RT\ln Q
\end{equation*}

When at equilibrium, $\Delta_rG=0$, so $\Delta_rG\zero=-RT\ln Q_{eq}$.
\begin{equation*}
    \Delta_rG\zero=\frac{\prod_p(\frac{P_p^{eq}}{P_0})^{\nu_p}}
    {\prod_r(\frac{P_r^{eq}}{P_0})^{\nu_r}}P\zero^{\Delta\nu}
\end{equation*}

$\Delta\nu$ is the change of stoichiometric coefficient in the reaction that
\begin{equation*}
    \Delta\nu = \sum_p\nu_p-\sum_r\nu_r
\end{equation*}

By convention, we define the equilibrium constant as
\begin{equation*}
    K_p=\frac{\prod_p(\frac{P_p^{eq}}{P_0})^{\nu_p}}
    {\prod_r(\frac{P_r^{eq}}{P_0})^{\nu_r}}P\zero^{\Delta\nu}
\end{equation*}
\subsection*{Variation of Equilibrium Constant with Temperature}
We know that
\begin{equation*}
    \ln K_p=-\frac{\Delta_rG\zero}{RT}
\end{equation*}

Differetiating the equation gives
\begin{equation*}
    \frac{\partial\ln K_p}{\partial T}=-\frac{1}{R}(\frac{\Delta_rG\zero/T}{\partial T})
\end{equation*}

And the Gibbs-Helmholtz equation shows that
\begin{equation*}
    \frac{\Delta_rG\zero/T}{\partial T}=-\frac{\Delta_rH\zero}{T^2}
\end{equation*}

So, we get
\begin{equation*}
    \frac{\partial\ln K_p}{\partial T}=-\frac{\Delta_rH\zero}{T^2}
\end{equation*}
\begin{equation*}
    \boxed{\frac{\diff\ln K_p}{\diff T}=\frac{\Delta_rH\zero}{RT^2}}
\end{equation*}

This is called van't Hoff equation.

For the change from $T\zero$ to $T$, we integrate the equation that
\begin{equation*}
    \ln K_p(T)-\ln K_p(T\zero)=\int_{T\zero}^{T}\frac{\Delta_rH\zero}{RT^2}}
\end{equation*}
\begin{equation*}
    \boxed{\ln K_p(T)=\ln K_p(T\zero)-\frac{\Delta_rH\zero}{R}(\frac{1}{T}-\frac{1}{T\zero})}
\end{equation*}
\end{document}