\item The radial equation for the hydrogen atom is:
\begin{align*}
    \left[-\frac{\hbar^2}{2m_e}\frac{1}{r^2}\pd{}{r}r^2\pd{}{r}-\frac{1}{4\pi\varepsilon_0}\frac{e^2}{r}+\frac{\hbar^2l(l+1)}{2m_er^2}\right]R(r)=E_nR_{nl}(r)
\end{align*}
The first three wavefunctions are:
\begin{align*}
    R_{10}(r) & =2\left(\frac{1}{a_0}\right)^{3/2}e^{-r/a_0}                                                           \\
    R_{20}(r) & =\left(\frac{1}{8}\right)^{1/2}\left(\frac{1}{a_0}\right)^{3/2}\left(2-\frac{r}{a_0}\right)e^{-r/2a_0} \\
    R_{21}(r) & =\left(\frac{1}{24}\right)\left(\frac{1}{a_0}\right)^{3/2}\frac{r}{a_0}e^{-r/2a_0}
\end{align*}

\begin{enumerate}
    \item Obtain an expression for $E_1$ in terms of ao and other constants by direct substitution of $R_{10}(r)$
          into the radial equation.
    \item Determine the most probable value(s) of r for the three radial functions.
    \item Determine the average value of both $r$ and $\frac{1}{r}$ for $R_{10}$.
    \item Determine $V(r)$ for $R_{10}$ and from $E_1$ determine $KE(r)$ for $R_{10}$.
    \item Calculate the spread in $r$ for $R_{10}$. What does this tell you about the momentum for a 1s electron?
\end{enumerate}

\begin{tcolorbox}
    \textbf{Radial distribution functions}
    \begin{align*}
        P_r(l=0) & =4\pi r^2R_n^2(r) \\
        P_r(l>0) & =r^2R_n^2(r)
    \end{align*}
\end{tcolorbox}

\begin{solution}\
    \begin{enumerate}
        \item Because this is 1s orbital, $l=0$, $m=0$, $n=1$. $\frac{\hbar^2l(l+1)}{2m_er^2}$ equals zero. The left hand side becomes
              \begin{align*}
                  LHS & = \left(-\frac{\hbar^{2}}{2 m_{e}}\frac{1}{r^{2}}\pd{}{r}r^{2}\pd{}{r}-\frac{1}{4\pi\varepsilon_{0}}
                  \frac{e^{2}}{r}\right)\left[2\left(\frac{1}{a_{0}}\right)^{3/2}e^{-r/a_{0}}\right]                                                                  \\
                      & =\left(-\frac{\hbar^{2}}{2 m_{e}}\cdot2\left(\frac{1}{a_{0}}\right)^{3/2}\frac{1}{r^{2}}\pd{}{r}r^{2}\pd{}{r}e^{-r/a_{0}}\right)-\frac{e^{2}}
                  {4\pi\varepsilon_{0}}\cdot2\left(\frac{1}{a_0}\right)^{3/2}\frac{1}{r}e^{-r/a_{0}}                                                                  \\
                      & =\frac{\hbar^{2}}{2m_e}\cdot2\left(\frac{1}{a_{0}}\right)^{3/2}\frac{1}{r^{2}}\pd{}{r}\left(r^{2}\frac{1}{a_{0}} e^{-r/a_{0}}\right)
                  -\frac{e^{2}}{4\pi\varepsilon_{0}}\cdot2\left(\frac{1}{a_{0}}\right)^{3/2}\frac{1}{r}e^{-r/a_{0}}                                                   \\
                      & =\frac{\hbar^{2}}{m_{e}}\left(\frac{1}{a_{0}}\right)^{3/2}\left(\frac{1}{a_{0}}\right)\frac{1}{r^{2}}\left(2re^{-r/a_{0}}-
                  \frac{r^{2}}{a_{0}} e^{-r/a_{0}}\right)-\frac{e^{2}}{2\pi\varepsilon_{0}}\left(\frac{1}{a_{0}}\right)^{3/2}\frac{1}{r}e^{-r/a_{0}}                  \\
                      & =\underbrace{2\left(\frac{1}{a_{0}}\right)^{3/2}e^{-r/a_{0}}}_{R_{10}(r)}
                  \underbrace{\left(\frac{\hbar^{2}}{a_{0}m_{e}}\frac{1}{r}-\frac{\hbar^{2}}{2a_{0}^{2}m_{e}}-\frac{e^{2}}{4\pi\varepsilon_{0}}\frac{1}{r}\right)}_{E_1}
              \end{align*}
              So,
              \begin{align*}
                  E_1=\frac{1}{r}\left(\frac{\hbar^{2}}{a_{0}m_{e}}-\frac{e^{2}}{4\pi\varepsilon_{0}}\right)-\frac{\hbar^{2}}{2a_{0}^{2} m_{e}}
              \end{align*}
        \item The most probable value is reached when $\pd{P(r)}{r}=0$. For $R_{10}$, $P(r)=4\pi r^{2}R_{10}^{2}$ because $l=0$.
              \begin{align*}
                  P(r)=4\pi r^{2}\left[2\left(\frac{1}{a_{0}}\right)^{3/2}e^{-r/a_{0}}\right]^{2}=4\pi r^{2}\cdot\frac{4}{a_{0}^{3}}e^{-2r/a_{0}}
              \end{align*}
              Then, take the derivative and let it equal to zero.
              \begin{align*}
                  \pd{P(r)}{r}=\frac{16\pi}{a_{0}^{3}}\left(2re^{-2r/a_{0}}-r^{2}\frac{2}{a_{0}}e^{-2r/a_{0}}\right)=0
              \end{align*}
              This further reduces to
              \begin{align*}
                  r\left(1-\frac{r}{a_0}\right)=0
              \end{align*}
              So $r_{mp}=a_0$ for $R_{10}$.

              For $R_{20}$, the form of radial distribution equation is the same.
              \begin{equation*}
                  P(r)=4\pi r^{2}\left[\left(\frac{1}{8}\right)^{1/2}\left(\frac{1}{a_{0}}\right)^{3/2}\left(2-\frac{r}{a_{0}}\right)
                  e^{-r/2a_{0}}\right]^{2}=4\pi r^{2}\frac{1}{8a_{0}^{3}}\left(\frac{2-r}{a_{0}}\right)^{2} e^{-r / a_{0}} \\
              \end{equation*}
              \begin{align*}
                  \pd{P(r)}{r} & =\pd{}{r}\left[\frac{\pi r^{2}}{2 a_{0}}\left(2-\frac{r}{a_{0}}\right)^{2}e^{-r/a_{0}}\right]             \\
                               & =\frac{\pi r}{a_{0}}\left(2-\frac{r}{a_{0}}\right)^{2}e^{-r/a_{0}}+\frac{\pi r^{2}}{2a_{0}}
                  \left(\frac{2r-4a_{0}}{a_{0}^{2}}\right)e^{-r/a_0}                                                                       \\
                               & \hspace{12pt}-\frac{\pi r^{2}}{2 a_{0}}\left(2-\frac{r}{a_{0}}\right)^{2}\frac{1}{a_{0}} e^{-r / a_{0}}=0
              \end{align*}
              This gives $r_{mp}=2a_0$ for $R_{20}$.

              For $R_{21}$, the radial distribution is $P(r)=r^2R_{21}^2$.
              \begin{align*}
                  P(r) & =r^{2}\left[\left(\frac{1}{24}\right)^{1/2}\left(\frac{1}{a_{0}}\right)^{3/2}\frac{r}{a_{0}}e^{-r/2a_{0}}\right]^{2} \\
                       & =r^{2}\frac{1}{24a_{0}^{3}}\frac{r^{2}}{a_{0}^{2}}e^{-r/a_{0}}
              \end{align*}
              \begin{align*}
                  \pd{P(r)}{r} & =\frac{1}{24a_{0}^{5}}\pd{}{r}\left(r^{4}e^{-r/a_{0}}\right)                          \\
                               & =\frac{1}{24a_{0}^{5}}\left[4r^{3}e^{-r/a_{0}}-\frac{r^{4}}{a_{0}}e^{-r/a_{0}}\right] \\
                               & =\frac{1}{24a_{0}^{5}}\left[r^{3}e^{-r/a_{0}}\left(4-\frac{r}{a_{0}}\right)\right]=0
              \end{align*}
              This gives $r_{mp}=4a_0$ for $R_{21}$.
        \item We know that $\langle A\rangle=\int\psi^*\hat{A}\psi\diff\tau$.
              \begin{align*}
                  \langle r\rangle & =\int_0^\infty R_{10}rR_{10}\underbrace{4\pi r^2\diff r}_{\text{volume element}} \\
                                   & =\int_0^\infty 4\pi r^2\cdot r\frac{4}{a_0^3}e^{-2r/a_0}\diff r                  \\
                                   & =\frac{16\pi}{a_0^3}\int_0^\infty r^3e^{-2r/a_0}\diff r                          \\
                                   & =\frac{16\pi}{a_0^3}\left(\frac{3a_0^4}{8}\right)=6\pi a_0
              \end{align*}
              And,
              \begin{align*}
                  \left\langle\frac{1}{r}\right\rangle & =\int_0^\infty R_{10}\frac{1}{r}R_{10}4\pi r^2\diff r                  \\
                                                       & =\frac{16\pi}{a_0^3}\int_0^\infty re^{-2r/a_0}\diff r=\frac{4\pi}{a_0}
              \end{align*}
        \item To find $V(r)$, we need to calculate $\langle V\rangle$ first. This is true for any observable.
              \begin{align*}
                  \langle V(r)\rangle & =\frac{e^2}{4\pi\varepsilon_0\langle r\rangle}
                  =\frac{e^2}{4\pi\varepsilon_0}\cdot\frac{4\pi}{a_0}                  \\
                                      & =\frac{e^2}{\varepsilon_0a_0}
              \end{align*}
              And for KE, $KE=\langle KE\rangle$.
              \begin{align*}
                  \langle KE\rangle & =E(r)-\langle V(r)\rangle                                                                       \\
                                    & =\frac{\hbar}{a_{0}m_{e}}\left\langle\frac{1}{r}\right\rangle\frac{-\hbar^{2}}{2a_{0}^{2}m_{e}} \\
                                    & =\frac{\hbar}{a_{0}m_{e}}\cdot\frac{4\pi}{a_{0}}\frac{-\hbar^{2}}{2a_{0}^{2}m_e}                \\
                                    & =\frac{8\pi\hbar-\hbar^2}{2a_{0}^{2}m_e}
              \end{align*}
        \item From part c, we know that $\langle r\rangle=6\pi a_0$.
              \begin{align*}
                  \langle r^2\rangle & =\int_{0}^{\infty}4\pi r^{2}\cdot r^{2}\cdot\frac{4}{a_{0}^{3}}e^{-2r/a_{0}}\diff r \\
                                     & =\frac{16\pi}{a_{0}^{3}}\int_{0}^{\infty}r^{4}e^{-2r/a_{0}}\diff r                  \\
                                     & =\frac{16\pi}{a_{0}^{3}}\cdot\frac{3 a^{5}}{4}=12\pi a_{0}^{2}
              \end{align*}
              \begin{align*}
                  \Delta r & =\left[\left\langle r^{2}\right\rangle-\langle r\rangle^{2}\right]^{1/2} \\
                           & =(12\pi a_0^2-36\pi^2 a_0^2)^{1/2}                                       \\
                           & =\sqrt{12\pi a_0^2(1-3\pi)}
              \end{align*}
              So, for a 1s electron, both the position and momentum are not exactly known.
    \end{enumerate}
\end{solution}