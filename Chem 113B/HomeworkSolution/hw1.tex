\item Solar energy strikes Earth's atmosphere at 343 W m$^{-2}$. About 30\% is reflected and the rest is
absorbed. The absorbed energy re-radiates as black body radiation at 5.67 $\times$ 10$^{-5}$ (T/K)$^4$ W m$^{-2}$,
where T is the temperature. Assuming this spectrum is at equilibrium, calculate the average black
body temperature of the Earth.  \\

\begin{solution}\ \\
    We can get the equation
    \begin{align*}
        343\times(1-30\%)=5.67\times 10^{-5}
    \end{align*}
    Solving the equation yields $T=45.4K$
\end{solution}

\item A certain metal has a work function of 3.1 eV. The metal absorbed $4.96\times 10^3$ J of light energy at
200 nm.
\begin{enumerate}
    \item How many electrons were emitted? Give your answer in absolute numbers.
    \item What is the kinetic energy of the electrons in eV?
    \item What is the de Broglie wavelength of these electrons in meters?
\end{enumerate}

\begin{solution}\
    \begin{enumerate}
        \item The energy absorbed by single electron is given by $E=hv$, so the total
              energy will be $E=Nhv$ where $N$ is the number of electrons.
              \begin{align*}
                  N=\frac{E}{hv}=4.5\times10^{21}
              \end{align*}
        \item The kinetic energy is total energy subtracted by work function.
              \begin{align*}
                  E_k=hv-\Phi=3.95\times10^{16}\mathrm{eV}
              \end{align*}
        \item The de Broglie wavelength is
              \begin{align*}
                  \lambda=\frac{h}{p}
              \end{align*}
              The momentum is given by
              \begin{align*}
                  E_k=\frac{p^2}{2m_e}
              \end{align*}
              This yieids $\lambda=2.97$ nm
    \end{enumerate}
\end{solution}

\item The Heisenberg uncertainty principle states: $\Delta A\Delta B\geq\hbar$
, where A and B are complimentary variables. Show $\Delta x\Delta p\geq\hbar/2$
for a particle in the box. The answer will be a function of n. Show it
holds for n = 1 and larger values of n.
\begin{tcolorbox}
    \textbf{Particle in the box}
    \begin{align*}
        \psi_n=\left(\frac{2}{L}\right)\sin\left(\frac{n\pi x}{L}\right)
    \end{align*}
\end{tcolorbox}

\begin{solution}\ \\
    The standard deviation is given by $\Delta x=[\langle x^2\rangle-
        \langle x\rangle^2]^{1/2}$.
    And the expected value of $x^2$ is
    \begin{align*}
        \langle x^2\rangle & =\int_0^L\psi^*x^2\psi \diff x
        =\frac{2}{L}\int_0^Lx^2\sin^2\left(\frac{n\pi x}{L}\right) \diff x       \\
                           & =\frac{2}{L}\left[\frac{L^3(4\pi^3n^3+(3-6\pi^2n^2)
                \sin(2\pi n)-6\pi n\cos(2\pi n))}{24\pi^3n^3}\right]
    \end{align*}
    Since n is integer, $\sin2\pi n=0$ and $\cos2\pi n=1$.
    \begin{align*}
        \left\langle x^{2}\right\rangle=\frac{L^{2}}{12 \pi^{3} n^{3}}\left(4 \pi^{3} n^{3}-6 \pi n\right)
        =L^{2}\left(\frac{1}{3}-\frac{1}{2 \pi^{2} n^{2}}\right)
    \end{align*}
    Similarly, for $x$, we have
    \begin{align*}
        \langle x\rangle & =\int_{0}^{L} \psi^{*} x \psi \diff x                                                   \\
                         & =\frac{2}{L} \int_{0}^{L} x \sin ^{2}\left(\frac{n \pi x}{L}\right) \diff x=\frac{L}{2}
    \end{align*}
    This also indicates the probability of finding the particle throughout the box is equal.
    And for momentum,
    \begin{align*}
        \langle p^{2}\rangle & =\int_{0}^{L} \psi^* \hbar^{2} \frac{\partial^{2} \psi}{\partial x^{2}} \diff x        \\
                             & =\frac{2}{L} \hbar^{2}\left(\frac{\pi^{2} n^{2}}{L^{2}}\right) \int_{0}^{L} \sin^{2}
        \left(\frac{n \pi x}{L}\right) \diff x                                                                        \\
                             & =\frac{2 \hbar^{2} \pi^{2} n^{2}}{L^{3}}\left[\frac{L}{4}\left(2-\frac{\sin (2 \pi n)}
            {\pi n}\right)\right]=\frac{\hbar^{2} \pi^{2} n^{2}}{L^{2}}
    \end{align*}
    \begin{align*}
        \langle p\rangle=\int_{0}^{L} \psi^{*}-i \hbar \frac{\partial}{\partial x} \psi \diff x=\frac{-2 i \hbar}{L}
        \int_{0}^{L} \sin \left(\frac{n \pi x}{L}\right)\cos \left(\frac{n \pi x}{L}\right) \diff x=0
    \end{align*}
    Thus,
    \begin{align*}
        \Delta x \Delta p & =\left[\left\langle x^{2}\right\rangle-\langle x\rangle^{2}\right]^{1 / 2}\left[\left\langle p^{2}
        \right\rangle-[p\rangle^{2}\right]^{1 / 2}                                                                                                      \\
                          & =L\left[\left(\frac{1}{3}-\frac{1}{2 \pi^{2} n^{2}}\right)-\frac{1}{4}\right]^{1 / 2}\left[\frac{\hbar \pi n}{\hbar}\right]
    \end{align*}
    For $n=1$: $\left(\frac{1}{3}-\frac{1}{2 \pi^{2}}-\frac{1}{4}\right)^{1/2}(\pi\hbar)=0.57\hbar\geq\frac{\hbar}{2}$.
    As $n$ increases, $\Delta x \Delta p$ gets larger, so all values of $n$ will support the Heisenberg uncertainty principle.
\end{solution}

\item Two unnormalized functions of the hydrogen atoms are:
\begin{align*}
    \psi_1=N_1\left(2-\frac{r}{a_0}\right)e^{-r/2a_0} \\
    \psi_2=N_2r\sin\theta\cos\phi e^{-r/2a_0}
\end{align*}
where $N_1$ and $N_2$ are the normalization constants.
\begin{enumerate}
    \item Normalize these functions.
    \item Show $\psi_1$ and $\psi_2$ are orthogonal.
\end{enumerate}

\begin{tcolorbox}
    \textit{The volume element for spherical polar coordinates is $r^2\diff r\sin\theta\diff\theta\diff\phi$. And the limits are
        $0\leq r\leq\infty$, $0\leq\theta\leq\pi$, and $0\leq\phi\leq2\pi$.}
\end{tcolorbox}

\begin{solution}\
    \begin{enumerate}
        \item The general equation for normalization is $\int\psi^*\psi\diff\tau=1$. And for $\psi_1$, this expands to
              \begin{align*}
                  N_{1}^{2}\int_{0}^{\infty}r^{2}\left(2-\frac{r}{a_{0}}\right)^{2}e^{-r/a_{0}}\diff r\int_{0}^{\pi}\sin\theta\diff
                  \theta\int_{0}^{2\pi}\diff\phi=1
              \end{align*}
              After solving the integral, the equation becomes
              \begin{align*}
                  \frac{1}{N_1^2}=8 a_{0}^{3}\cdot 2\cdot 2 \pi
              \end{align*}
              So, $N_1=\frac{1}{4 \sqrt{2 \pi} a_{0}^{3 / 2}}$.

              Similarly,
              \begin{align*}
                  N_{2}^{2}\int_{0}^{\infty}r^{3}e^{-r/a_{0}}\diff r\int_{0}^{\pi}\sin^{3}\theta\diff\theta\int_{0}^{2\pi}
                  \cos^{2}\phi\diff\phi=1
              \end{align*}
              And $N_2=\frac{1}{\sqrt{8\pi} a_{0}^{2}}$.
        \item If $\psi_1$ and $\psi_2$ are orthogonal,$\int\psi_1^*\psi_2\diff\tau=0$. In this case, it equals
              \begin{align*}
                  \left(\frac{1}{4\sqrt{2\pi}a_{0}^{3/2}}\right)\left(\frac{1}{\sqrt{8\pi}a_{0}^{2}}\right)\int_{0}^{\infty}
                  r^{2}\left(2-\frac{r}{a_{0}}\right)e^{-r/a_{0}}\diff r\int_{0}^{\pi}\sin^{2}\theta\diff\theta\int_{0}^{2\pi}\cos\phi\diff\phi
              \end{align*}
              Because $\int_{0}^{2\pi}\cos\phi\diff\phi=0$, the whole thing becomes zero, so the two wavefunctions are orthogonal.
    \end{enumerate}
\end{solution}

\item \begin{enumerate}
    \item Evaluate the commutators $[\hat{H},\hat{p_x}]$ and $[\hat{H},\hat{x}]$ for the cases
          \begin{enumerate}
              \item $V(x)=\text{constant}$
              \item $V(x)=f(x)$
          \end{enumerate}
    \item For the free particle, $V(x)=\text{constant}$. What do the results in (a) part i. tell us about $\hat{H}$ and $\hat{p_x}$?
          Is this general result found in our treatment of the free particle? Discuss and justify your
          answer.
    \item For the particle in a box, $V(x)=f(x)$. In this case, what do the results in a. part ii. tell us about
          $\hat{H}$ and $\hat{p_x}$? Is this general result found in our treatment of the particle in the box? Discuss and
          justify your answer.
\end{enumerate}

\begin{solution}\
    \begin{enumerate}
        \item \begin{enumerate}
                  \item \begin{align*}
                            [\hat{H},\hat{p_x}] & =\hat{H}\hat{p_x}\psi-\hat{p_x}\hat{H}\psi                                       \\
                                                & =\hat{KE}\hat{p_x}\psi+V(x)\hat{p_x}\psi-\hat{p_x}\hat{KE}\psi-\hat{p_x}V(x)\psi \\
                                                & =0
                        \end{align*}
                        \begin{align*}
                            [\hat{H},\hat{x}] & =\hat{H}\hat{x}\psi-\hat{x}\hat{H}\psi                            \\
                                              & =\hat{KE}x\psi+V(x)x\psi-x\hat{KE}\psi-xV(x)\psi                  \\
                                              & =-\frac{\hbar^2}{2m}\pd{\psi}{x}-\frac{\hbar^2}{2m}x\pdd{\psi}{x}
                            +\frac{x\hbar^2}{2m}\pdd{\psi}{x}                                                     \\
                                              & =-\frac{\hbar^2}{2m}\pd{}{x}
                        \end{align*}
                  \item \begin{align*}
                            [\hat{H},\hat{p_x}] & =\hat{KE}\hat{p_x}\psi+V(x)\hat{p_x}\psi-\hat{p_x}\hat{KE}\psi-
                            (\hat{p_x}V(x)\psi+V(x)\hat{p_x}\psi)                                                 \\
                                                & =i\hbar\pd{V(x)}{x}
                        \end{align*}
                        \begin{align*}
                            [\hat{H},\hat{x}] & =\hat{KE}x\psi+V(x)x\psi-x\hat{KE}\psi-xV(x)\psi                     \\
                                              & =(-\frac{\hbar^2}{2m}\pd{\psi}{x}-\frac{\hbar^2}{2m}x\pdd{\psi}{x})+
                            \frac{x\hbar^2}{2m}\pdd{\psi}{x}                                                         \\
                                              & =-\frac{\hbar^2}{2m}\pd{}{x}
                        \end{align*}
              \end{enumerate}
        \item For the free particle, $\hat{H}$ and $\hat{p_x}$ commute ($V(x)$ is a constant). This makes sense since
              the kinetic energy portion of the Hamiltonian is just $\frac{p^2}{2m}$. If we know $p$ exactly, of course we'd
              also know $p^2$ exactly.
        \item For the particle in a box, $\hat{H}$ and $\hat{p_x}$ do not commute, because the potential term depends
              on position. As per the Heisenberg uncertainty principle, position and momentum cannot be simultaneously
              known (they do not commute). Since the Hamiltonian now contains a position-dependent term, it can not longer
              commute with the momentum.
    \end{enumerate}
\end{solution}