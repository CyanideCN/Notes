\documentclass[letterpaper]{article}
\usepackage[utf8]{inputenc}
\usepackage{amsmath}
\usepackage{lmodern}
\usepackage[margin=1in]{geometry}
\usepackage{indentfirst}
\usepackage{multirow}

\newcommand{\pd}[2]{\frac{\partial #1}{\partial #2}}
\newcommand{\pdd}[2]{\frac{\partial^2 #1}{\partial #2^2}}
\newcommand{\diff}{\mathrm{d}}

\makeatletter
\begin{document}
\newpage
\section*{Chapter 10 - Molecular Symmetry}
\subsection*{Group theory}
Common elements in group theory.
\begin{table}[!htb]
    \begin{align*}
        \begin{tabular}{|c|c|c|}
            \hline
            \textbf{Element}                     & \textbf{Symbol}  & \textbf{Function}                                \\
            \hline
            Identity                             & $\hat{E}$        & No change                                        \\
            \hline
            Rotation                             & $\hat{C_n}$      & Rotation by $2\pi / n$                           \\
            \hline
            \multirow{3}{*}{Reflection in plane} & $\hat{\sigma_v}$ & Principal axis                                   \\
            \cline{2-3}
                                                 & $\hat{\sigma_n}$ & Perpendicular to principal axis                  \\
            \cline{2-3}
                                                 & $\hat{\sigma_d}$ & Bisects principal axis                           \\
            \hline
            Inversion                            & $\hat{i}$        & Inversion at center of symmetry                  \\
            \hline
            Improper rotation                    & $\hat{S_n}$      & Rotation of $2\pi/n$ follows by $\hat{\sigma_n}$ \\
            \hline
        \end{tabular}
    \end{align*}
\end{table}

\subsubsection*{Rules}
\begin{enumerate}
    \item Associative law: $(\hat{A}\hat{B})\hat{C}=\hat{A}(\hat{B}\hat{C})$.
    \item All groups contain $\hat{E}$.
    \item $\hat{A}^{-1}\hat{A}=\hat{E}$ for all operations.
    \item Groups are composed of class of equivalent elements.
    \item Groups have irreducible representations (IR) that form the basic set of the group.
          \begin{enumerate}
              \item Number of IRs is the number of elements in the group.
              \item Character table is the table of IRs.
              \item Reducible representations are composed of linear combinations of IRs.
              \item Individual symmetry characters of IRs are obtained from  the traces of the matrix
                    of the symmetry operator.
              \item All member of the same class have the same characters.
          \end{enumerate}
\end{enumerate}

\subsubsection*{Character table}
Tables showing all the characters of the operations of a group are called character tables . The columns of a character table are
labelled with the symmetry operations of the group. Although the notation $\Gamma$ is used to label general irreducible representations,
in chemical applications and for displaying character tables it is more common to distinguish different irreducible representations by the use of the labels A, B, E, and T to denote
the symmetry species of each representation:
\begin{itemize}
    \item A: one-dimensional representation, character +1 under the principal rotation.
    \item B: one-dimensional representation, character -1 under the principal rotation.
    \item E: two-dimensional irreducible representation.
    \item T: three-dimensional irreducible representation.
\end{itemize}
Subscripts are used to distinguish the irreducible representations if there is more than one of the same type.

\subsubsection*{$C_{3V}$ group}
The planar triangle belongs to the $C_{3V}$ group.
\end{document}